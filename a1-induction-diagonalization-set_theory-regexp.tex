% Copyright (C) 2011  Dennis Ideler (dennisideler.com)
%    Permission is granted to copy, distribute and/or modify this document
%    under the terms of the GNU Free Documentation License, Version 1.3
%    or any later version published by the Free Software Foundation;
%    with no Invariant Sections, no Front-Cover Texts, and no Back-Cover Texts.
%    A copy of the license is included in the section entitled "GNU
%    Free Documentation License" at http://www.gnu.org/copyleft/fdl.html

\documentclass[10pt,a4paper,final]{article}
\usepackage[latin1]{inputenc}
\usepackage{amsmath}
\usepackage{amsfonts}
\usepackage{amssymb}
\usepackage{graphicx}
\setlength{\topmargin}{-.5in}
\setlength{\textheight}{9in}
\setlength{\oddsidemargin}{.125in}
\setlength{\textwidth}{6.25in}
\author{Dennis Ideler}
\title{MATH/COSC 4P61: Theory of Computation\\Assignment 1}
\begin{document}
\maketitle

\begin{enumerate}
\item % Q1
Prove by induction that for $n \geq 4$ the statement $S(n) = n! > 2^n$ is true.\\
\\
Proof by induction has two steps: proving the base case and the inductive step.
These two parts should convince us that $S(n)$ is true for every integer $n$
that is equal to or greater than the basis integer $4$.\\
\\
\textbf{Step 1: Basis}\\ % Prove base case
$S(4) = 4! > 2^4 = 24 > 16$.\\
\\
\textbf{Step 2: Induction}\\
Inductive hypothesis: assume true for $k \geq 4$.\\
If $S(k) = k! > 2^k$ then $S(k+1) = (k+1)! > 2^{k+1}$.\\
Rewrite $S(k+1)$ so it can make use of $S(k)$.\\
$S(k+1) = (k+1)\,k! > 2 \cdot 2^k$\\
$S(n)$ tells us that $k! > 2^k$.
If we remove that assumed truth from the above statement, we only need to show $k+1 > 2$.
Since $n \geq 4$, we get $4 + 1 > 2$ which holds.\\
\\
$\therefore n! > 2^n$ for $n  \geq 4$.

\item % Q2
A function $f : N \rightarrow N$ is monotone-increasing if $f(i) < f(i+1) \: \forall \: i \in N$.\\
Prove, using diagonalization, that the set of monotone-increasing functions is
uncountable.\\
\\
%Say there is a table that enumerates all subsets for some assumed countable set.
%Diagonalization is the trick of complementing the diagonal to construct a subset
%that cannot be a subset that appears in any row of the table.
%It works because the complement of the diagonal is itself a characteristic vector describing
%membership in some set. This characteristic vector disagrees in some column with every row of the
%suggested table.\\
An uncountable set is an infinite set that contains too many elements to be countable
(i.e. it is not enumerable).
The uncountability of a set is closely related to its cardinal number:
a set is uncountable if its cardinal number is larger than that of the set of all natural numbers.
The set of natural numbers (and any other countably infinite set) has cardinality aleph-null
($\aleph_0$).\\
\\
Technique of Diagonalization\\
Step 1: Assume set is countable and enumerate all subsets\\
Step 2: Create new subset utilizing elements from existing subsets\\
Step 3: Show new subset is in set but not in enumeration\\
\\
1) We assume the set of all monotone-increasing functions is countable
\begin{tabular}{c|c c c c c}
& 1 & 2 & 3 & $\cdots$ & $i$ \\
\hline
$f_1$ & \textbf{1} & 2 & 3 & $\cdots$\\
$f_2$ & 2 & \textbf{4} & 6 & $\cdots$\\
$f_3$ & 3 & 5 & \textbf{7} & $\cdots$\\
$\vdots$\\
$f_i$\\
\end{tabular}
2) Create a new monotone-increasing function $f(i) = f(i-1) + f_i(i),\: f \neq f_i \: \forall \: i$.\\
$f(1) = 0 + f_1(1) = 0 + 1 = 1\\
f(2) = 1 + f_2(2) = 1 + 4 = 5\\
f(3) = 5 + f_3(3) = 5 + 7 = 12\\
\vdots$\\
\\
3) $f(i) \in N$ but not in enumeration (differs in some column with every row).\\
$\therefore$ the set of monotone-increasing functions is uncountably infinite.

\item % Q3
Prove, using diagonalization, that the power set of natural numbers ($2^N$) is uncountable.\\
\\
Step 1: Assume $2^N$ is countable, enumerate all subsets\\
Step 2: Create new subset utilizing existing subsets\\
Step 3: Show new subset is in power set but not in enumeration\\
\\
1) We use binary representation for the subsets (easier to enumerate)
\begin{tabular}{c|c c c c c}
& 1 & 2 & 3 & $\cdots$ & $N$ \\
\hline
$S_1$ & \textbf{0} & 0 & 0 & $\cdots$\\
$S_2$ & 1 & \textbf{1} & 1 & $\cdots$\\
$S_3$ & 1 & 0 & \textbf{1} & $\cdots$\\
$\vdots$\\
$S_i$\\
\end{tabular}\\
2) Create a new subset $S(i) = 1-S_i(i), \: S \neq S_i \: \forall \: i$.\\
$S(1) = 1 - S_1(1) = 1 - 0 = 1\\
S(2) = 1 - S_2(2) = 1 - 1 = 0\\
S(3) = 1 - S_3(3) = 1 - 1 = 0\\
\vdots$\\
\\
3) $S(i) \in 2^N$ (since the power set contains all subsets of set $N$) but not in enumeration.\\
$\therefore$ the power set of all natural numbers is uncountably infinite.

\item % Q4
We know the set of all functions $f:N \rightarrow \{0,1\}$ is uncountable.\\
What about the set of all functions $f:\{0,1\} \rightarrow N$?\\
\\
This function has two cases: $f_i(0) \rightarrow N$ and $f_i(1) \rightarrow N$.\\
We know that $|N| = \aleph_0$ which means the set of $N$ is countable.\\
We know $f:\{0,1\} \rightarrow N$ is equivalent to
$f:\{0\} \rightarrow N \, \bigcup \, f:\{1\} \rightarrow N$.\\
The union of two countable sets results in a countable set, $N \bigcup N = N$.\\
$\therefore$ the set of all functions $f:\{0,1\} \rightarrow N$ is countably infinite.

% TODO
\item % Q5
One-to-one and onto functions:\\
$A$ onto $B \Rightarrow$ every element in $B$ is mapped.\\
$A$ 1-1 $B \Rightarrow$ no element in $B$ maps to more than one element in $A$.\\
1-1 correspondence $\Rightarrow$ bijective (1-1 \emph{and} onto).
\emph{i.e.} No two values map to the same value.
For every element in the codomain, some element of the domain maps to it.\\
Identity function $\Rightarrow$ input parameter is the same as the output value.\\
\\
Find functions $f:N \rightarrow N$ where $N = \{1,2,3,\dots\}$ that satisfy
\begin{enumerate}
  \item $f$ is 1-1 but not onto:\\
  $f(x) = x + 2$\\
  $f(1) = 3$\\
  $f(2) = 4$\\
  $f(3) = 5$\\
  $f(4) = 6$\\
  $f(5) = 7$\\
  $f(6) = 8$\\
  $\vdots$\\
  \item $f$ is onto but not 1-1:\\
  $f(x) = \lceil \frac{x}{2} \rceil$\\
  $f(1) = 1$\\
  $f(2) = 1$\\
  $f(3) = 2$\\
  $f(4) = 2$\\
  $f(5) = 3$\\
  $f(6) = 3$\\
  $\vdots$\\
  \item $f$ is a 1-1 correspondence and $f$ is not an identity function:\\
  $f(x) = x - (x + 2 \: mod \: 3) + (x \: mod \: 3)$\\
  $f(1) = 2$\\
  $f(2) = 3$\\
  $f(3) = 1$\\
  $f(4) = 5$\\
  $f(5) = 6$\\
  $f(6) = 4$\\
  $\vdots$\\
  A conditional function would also work and would be more simple.
\end{enumerate}

\item % Q6
Let $X = \{aa, bb\}$ and $Y = \{\epsilon, b, ab\}$.
\begin{enumerate}
  \item List the strings in the set $XY$.\\ % XY = X \cdot Y
  $XY = X \cdot Y = \{aa, bb\}\{\epsilon, b, ab\}$
  \begin{enumerate}
    \item $aa \epsilon = aa$
    \item $aa b$
    \item $aa ab$
    \item $bb \epsilon = bb$
    \item $bb b$
    \item $bb ab$
  \end{enumerate}
  \item List the strings of the set $Y^*$ of length three or less.\\
  $Y^* = \{\epsilon, b, ab\}^*$ of length 3 or less.
  \begin{enumerate}
  	\item $|\epsilon| = 0$
  	\item $|b| = 1$
  	\item $|b b| = 2$
  	\item $|b b b| = 3$
  	\item $|ab| = 2$
  	\item $|b ab| = 3$
  	\item $|ab b| = 3$
  \end{enumerate}
  \item How many strings of length 6 are there in $X^*$?\\
  $X^* = \{aa,bb\}^*$ of length 6\\
  Each symbol has length 2, so there have to be 3 symbols in each string. $2^3 = 8$ strings.
  \begin{enumerate}
  	\item $000 \rightarrow aa aa aa$
 	\item $001 \rightarrow aa aa bb$
  	\item $010 \rightarrow aa bb aa$
  	\item $011 \rightarrow aa bb bb$
  	\item $100 \rightarrow bb aa aa$
  	\item $101 \rightarrow bb aa bb$
  	\item $110 \rightarrow bb bb aa$
  	\item $111 \rightarrow bb bb bb$
  \end{enumerate}
\end{enumerate}

\item % Q7
Let $L_1 = \{aaa\}^*$, $L_2 = \{a,b\}\{a,b\}\{a,b\}\{a,b\}$, and $L_3 = L_2^*$.
Describe the strings that are in the languages
\begin{enumerate}
  \item $L_2 =$ strings of length 4 that are any combination of $a$'s and $b$'s.
  \item $L_3 =$ closure of $L_2$, 0 or more occurrences of $L_2$.
  i.e. empty string and strings that are a multiple of 4 with any combination of $a$'s and $b$'s.
  E.g. $\epsilon$, $aaab$, $bbbbaaaa$, etc.
  \item $L1 \bigcap L_3 =$ intersection of $L_1$ and $L_3$, which are strings of symbol $aaa$
  that are multiples of 12. $L1 \bigcap L_3 = \{aaa\,aaa\,aaa\,aaa\}^*$
\end{enumerate}
\end{enumerate}
\end{document}
