% Copyright (C) 2011  Dennis Ideler (dennisideler.com)
%    Permission is granted to copy, distribute and/or modify this document
%    under the terms of the GNU Free Documentation License, Version 1.3
%    or any later version published by the Free Software Foundation;
%    with no Invariant Sections, no Front-Cover Texts, and no Back-Cover Texts.
%    A copy of the license is included in the section entitled "GNU
%    Free Documentation License" at http://www.gnu.org/copyleft/fdl.html

\documentclass[10pt,a4paper,final]{article}
\usepackage[latin1]{inputenc}
\usepackage{amsmath}
\usepackage{amsfonts}
\usepackage{amssymb}
\usepackage{graphicx}
\usepackage{upgreek} % for /Upphi
\setlength{\topmargin}{-.5in}
\setlength{\textheight}{9in}
\setlength{\oddsidemargin}{.125in}
\setlength{\textwidth}{6.25in}
\author{Dennis Ideler}
\title{MATH/COSC 4P61: Theory of Computation\\Assignment 3}
\begin{document}
\maketitle

\begin{enumerate}
\item % Q1
Find context-free grammars\footnote{Context-free grammars are important in computer science
for describing the structure of programming languages and other artificial languages.}
(CFG's) for the following languages:
\begin{enumerate}
  \item % a
  The set of all production rules for CFG's with $T = \Sigma = \{a,b\}$ and $V = \{A,B,C\}$.
  \begin{eqnarray*}
  S &\rightarrow& A \,|\, B \,|\, C \\
  A &\rightarrow& aA \,|\, aB \,|\, \epsilon \\ % aB and bA not necessary, but is it wrong to include?
  B &\rightarrow& bB \,|\, bA \,|\, \epsilon \\
  C &\rightarrow& aC \,|\, bC \,|\, \epsilon
  \end{eqnarray*}
  
  \item % b
  The set of all strings over alphabet $\{a,b,\cdot,+,^*,(,),\underline{\epsilon},\Upphi\}$
  that are well-formed regular expressions over alphabet $\{a,b\}$. Note that we must distinguish
  between $\epsilon$ as the empty string and as a symbol in a regular expression. We use
  $\underline{\epsilon}$ in the latter case.
  \begin{eqnarray*}
  S &\rightarrow& a \,|\, b \,|\, S \cdot S \,|\, S + S \,|\, S^* \,|\, (S) \,|\,
   \underline{\epsilon} \,|\, \Upphi
  \end{eqnarray*}
  
  \item % c
  The set of all strings over alphabet $\{a,b\}$ such that if $w$ is in the set, then $3\,|\,|w|$
  (i.e. 3 divides the cardinality of $w$, or, $w$ is a multiple of 3, $w$ can be 0).
  \begin{eqnarray*}
  S &\rightarrow& \epsilon \,|\, aaaS \,|\, aabS \,|\, abaS \,|\, abbS \,|\, baaS \,|\, babS \,|\,
   bbaS \,|\, bbbS
  \end{eqnarray*}
  
  \item % d
  $\{a^i b^j c^k | i \neq j$ or $j \neq k\}$ \\
  \\
  Because $i \neq j$ \textbf{\emph{or}} $j \neq k$, there will be 4 types of strings. \\
  If $i \neq j$, then $i > j$ or $i < j$.
  If $j \neq k$, then $j > k$ or $j < k$.\\
  In my CFG, $ABC$ is for $i > j$, $DEF$ is for $i < j$, $GHI$ is for $j > k$, and
  $JKL$ is for $j < k$.
  \begin{eqnarray*}
  S &\rightarrow& ABC \,|\, DEF \,|\, GHI \,|\, JKL \\
  \\
  A &\rightarrow& aA \,|\, a \\
  B &\rightarrow& aBb \,|\, ab \,|\, \epsilon \\
  C &\rightarrow& cC \,|\, c \,|\, \epsilon \\
  \\
  D &\rightarrow& aAb \,|\, ab \,|\, \epsilon \\
  E &\rightarrow& bB \,|\, b \\
  F &\rightarrow& cC \,|\, c \,|\, \epsilon \\
  \\
  G &\rightarrow& aA \,|\, a \,|\, \epsilon \\
  H &\rightarrow& bB \,|\, b \\
  I &\rightarrow& bCc \,|\, bc \,|\, \epsilon \\
  \\
  J &\rightarrow& aA \,|\, a \,|\, \epsilon \\
  K &\rightarrow& bBc \,|\, bc \,|\, \epsilon \\
  L &\rightarrow& cC \,|\, c
  \end{eqnarray*}
\end{enumerate}

\item % Q2
Let $G$ be the grammar
\begin{eqnarray}
  S &\rightarrow& ASB \\
  S &\rightarrow& ab \\
  S &\rightarrow& SS \\
  A &\rightarrow& aA \\
  A &\rightarrow& \epsilon \\
  B &\rightarrow& bB \\
  B &\rightarrow& \epsilon
\end{eqnarray}

\begin{enumerate}
  \item % a
  Give a leftmost derivation\footnote{A derivation of a string for a grammar is a sequence of grammar
  rule applications, that transforms the start symbol into the string. A derivation proves that the
  string belongs to the grammar's language. In a leftmost derivation, the next nonterminal to rewrite
  is always the leftmost nonterminal; in a rightmost derivation, it is always the rightmost
  nonterminal.} of $aaabb$.
  \begin{eqnarray}
    \setcounter{equation}{1}
    S &\Rightarrow& ASB \\
    \setcounter{equation}{4}
    S &\Rightarrow& aASB \\
    \setcounter{equation}{4}
    S &\Rightarrow& aaASB \\
    \setcounter{equation}{5}
    S &\Rightarrow& aa \epsilon SB = aaSB \\
    \setcounter{equation}{2}
    S &\Rightarrow& aaabB \\
    \setcounter{equation}{6}
    S &\Rightarrow& aaabbB \\
    \setcounter{equation}{7}
    S &\Rightarrow& aaabb\epsilon = aaabb
  \end{eqnarray}

  \item % b
  Give a rightmost derivation of the same string.
  \begin{align} % NOTE: eqnarray is discouraged (spacing issues), so trying align instead...
    \setcounter{equation}{0} % NOTE: setcounter works differently in align (-1)
    S &\Rightarrow ASB \\
    \setcounter{equation}{5}
    S &\Rightarrow ASbB \\
    S &\Rightarrow ASb \epsilon = ASb \\
    \setcounter{equation}{1}
    S &\Rightarrow Aabb \\
    \setcounter{equation}{3}
    S &\Rightarrow aAabb \\
    \setcounter{equation}{3}
    S &\Rightarrow aaAabb \\
    S &\Rightarrow aa \epsilon abb = aaabb
  \end{align}

  \item % c
  Show that $G$ is ambiguous\footnote{A context-free grammar is said to be an ambiguous grammar if there
  exists a string that can be generated by the grammar in more than one way (i.e. the string admits more
  than one parse tree or, equivalently, more than one leftmost derivation).}. \\
  \\
  It was already shown in 2a and 2b that more than one parse tree exists for the same string, which
  means $G$ is ambiguous. Another example to show it is, can be with the string $ab$.
  \begin{align}
    \setcounter{equation}{1} % NOTE: setcounter works differently in align (-1)
    S &\Rightarrow ab
  \end{align}
  or
  \begin{align}
    \setcounter{equation}{0}
    S &\Rightarrow ASB \\
    \setcounter{equation}{4}
    S &\Rightarrow \epsilon SB = SB \\
    \setcounter{equation}{6}
    S &\Rightarrow S \epsilon = S \\
    \setcounter{equation}{1}
    S &\Rightarrow ab
  \end{align}

  \item % d
  Give an unambiguous grammar equivalent to $G$. \\
  \\
  Some ambiguous grammars can be converted into unambiguous grammars, but no general procedure
  for doing this is possible just as no algorithm exists for detecting ambiguous grammars.
  So instead we describe what type of strings $G$ produces and then try to find a grammar that
  (1) is equivalent to $G$ (i.e. generates the same language), and (2) is unambiguous
  (i.e. for every sentence of the language, the parse tree is correct).\\
  \\
  $G$ produces strings that start with an $a$ and end with a $b$ and have any combination of $a$'s and
  $b$'s in between.
  \begin{align*}
    S &\Rightarrow aTb \\
    T &\Rightarrow aT \,|\, bT \,|\, \epsilon
  \end{align*}
\end{enumerate}

\item % Q3
Show that the grammar $S \rightarrow aSbS|bSaS|\epsilon$ is ambiguous. \\
\\
First we number the production rules so we can reference them in our derivation.
\begin{align}
  \setcounter{equation}{0}
  S &\Rightarrow aSbS \\ % 1
  S &\Rightarrow bSaS \\ % 2
  S &\Rightarrow \epsilon % 3
\end{align}
For the string $babaab$ we can find two different derivations and associated parse trees.
\begin{align}
  \setcounter{equation}{1}
  S &\Rightarrow bSaS \\
  \setcounter{equation}{0}
  S &\Rightarrow b aSbS aS \\
  \setcounter{equation}{2}
  S &\Rightarrow b a \epsilon bS aS = b abS aS\\
  \setcounter{equation}{2}
  S &\Rightarrow b ab \epsilon aS = b ab aS \\
  \setcounter{equation}{0}
  S &\Rightarrow baba aSbS \\
  \setcounter{equation}{2}
  S &\Rightarrow baba a \epsilon bS = babaabS\\
  \setcounter{equation}{2}
  S &\Rightarrow babaab \epsilon = babaab
\end{align}
Which has the parse tree
% TODO draw on paper
\\ \\ \\ \\ \\ \\ \\ \\ \\ \\ \\ \\ \\ \\ \\ \\ \\
and
\begin{align}
  \setcounter{equation}{1}
  S &\Rightarrow bSaS \\
  \setcounter{equation}{2}
  S &\Rightarrow b \epsilon aS = baS \\
  \setcounter{equation}{1}
  S &\Rightarrow ba bSaS \\
  \setcounter{equation}{2}
  S &\Rightarrow bab \epsilon aS = babaS \\
  \setcounter{equation}{0}
  S &\Rightarrow baba aSbS \\
  \setcounter{equation}{2}
  S &\Rightarrow babaa \epsilon bS = babaabS \\
  \setcounter{equation}{2}
  S &\Rightarrow babaab \epsilon = babaab
\end{align}
Which has the parse tree
% TODO draw on paper
\\ \\ \\ \\ \\ \\ \\ \\ \\ \\ \\ \\ \\ \\ \\ \\ \\

\item % Q4
Construct push-down automata (PDA's) for the following languages and state whether it is
``accept by empty stack'' or ``accept by final state.''

\begin{enumerate}
  \item % a
  $\{w|w \in (a+b)^+$ and contains the same number of $a$'s and $b$'s$\}$  \\
  \\
  We match the $a$'s and $b$'s by pushing and popping.
  If the stack ever becomes empty, we have the same numbers of $a$'s and $b$'s.
  We will accept this language by empty stack, so the stack will start out with empty stack
  symbol $Z_0$ and finish with no symbols.
  \\
  \begin{align*}
    M &= \{Q, \Sigma, \Gamma, \delta, q_0, Z_0, F\} \\
    Q &= \{q_0\} \\
    \Sigma &= \{a, b\} \\
    \Gamma &= \{Z_0, A, B\} \\
    \delta(q_0, a, Z_0) &= \{(q_0, AZ_0)\} \mbox{ `a' encountered on empty stack, push A} \\
    \delta(q_0, b, Z_0) &= \{(q_0, BZ_0)\} \mbox{ `b' encountered on empty stack, push B} \\
    \delta(q_0, a, A) &= \{(q_0, AA)\} \mbox{ keep pushing A} \\
    \delta(q_0, b, B) &= \{(q_0, BB)\} \mbox{ keep pushing B} \\
    \delta(q_0, a, B) &= \{(q_0, \epsilon)\} \mbox{ `a' encountered, pop B} \\
    \delta(q_0, b, A) &= \{(q_0, \epsilon)\} \mbox{ `b' encountered, pop A} \\
    \delta(q_0, \epsilon, Z_0) &= \{(q_0, \epsilon)\} \mbox{ end of input, pop special stack symbol}
  \end{align*}
%  In state $q_0$ for each symbol $a$ read, one $A$ is pushed onto the stack. \\
%  Pushing symbol $A$ on top of the empty stack $Z_0$ is formalized as replacing top $Z_0$ by $AZ_0$. \\
%  Pushing symbol $A$ on top of another $A$ is formalized as replacing top $A$ by $AA$. \\
%  \\
%  In state $q_1$ for each symbol $b$ read one $A$ is popped. \\
%  At any moment the automaton may move from state $q_0$ to state $q_1$. \\
%  It may only accept by empty stack when from state $q_1$ when the stack consists of a single $Z_0$. \\
  
  \item % b
  $\{ww^R|w \in (a+b)^+\}$ \\
  \\
  This PDA will recognize the language of even length palindromes (with length greater than 0)
  over $\Sigma = \{a,b\}$. This PDA pushes the input symbols on the stack until it guesses that it is
  in the middle and then it compares the input with what is on the stack, popping off symbols from the
  stack as it goes. If it reaches the end of the input precisely at the time when the stack is empty,
  it accepts. Thus it accepts by empty stack, and it is non-deterministic. To ensure it is a valid
  palindrome, we use two states, $q_0$ for pushing and $q_1$ for popping.
  \begin{align*}
    M &= \{Q, \Sigma, \Gamma, \delta, q_0, Z_0, F\} \\
    Q &= \{q_0, q_1\} \\
    \Sigma &= \{a, b\} \\
    \Gamma &= \{Z_0, A, B\} \\
    \delta(q_0, a, Z_0) &= \{(q_0, AZ_0)\} \mbox{ push A} \\
    \delta(q_0, b, Z_0) &= \{(q_0, BZ_0)\} \mbox{ push B} \\
    \delta(q_0, a, A) &= \{(q_0, AA), (q_1, \epsilon)\} \mbox{ push or pop A} \\
    \delta(q_0, b, B) &= \{(q_0, BB), (q_1, \epsilon)\} \mbox{ push or pop B} \\
    \delta(q_0, a, B) &= \{(q_0, AB), (q_1, \epsilon)\} \mbox{ push or pop A} \\
    \delta(q_0, b, A) &= \{(q_0, BA), (q_1, \epsilon)\} \mbox{ push or pop B} \\
    \delta(q_1, a, A) &= \{(q_1, \epsilon)\} \mbox{ pop A} \\
    \delta(q_1, b, B) &= \{(q_1, \epsilon)\} \mbox{ pop B} \\
    \delta(q_1, a, B) &= \{(q_1, \epsilon)\} \mbox{ pop A} \\
    \delta(q_1, b, A) &= \{(q_1, \epsilon)\} \mbox{ pop B} \\
    \delta(q_1, \epsilon, Z_0) &= \{(q_1, \epsilon)\} \mbox{ end of input, pop special stack symbol}
  \end{align*}
\end{enumerate}
  
\item % Q5
Which of the following languages over $\Sigma = \{a,b\}$ is a context-free language (CFL)
(i.e. a language generated by a CFG)? Prove your answers.
If CFL, give a PDA or CFG. If not CFL, show with pumping lemma.
\begin{enumerate}
  \item % a
  The set of strings of $a$'s, $b$'s, and $c$'s, with an equal number of each. \\
  \\
  Intuition tells us it is not CFL because you would need to count, which we cannot do.
  So we should with pumping lemma that it is not CFL.
  There are many cases where it will violate the language property, we just need one.
  \begin{align*}
    \mbox{Let } z &= a^n b^n c^n \\
    &= u v^i w x^i y \\
    |vwx| \leq n \\
    |vx| \geq 1 \\
    v,x \mbox{ contains } a's: \\
    \overbrace{\underbrace{a \cdots a}_{uvwxy}}^{n} \overbrace{b \cdots b}^{n} \overbrace{c \cdots c}^{n} \\
    u v^i w x^i y \notin L \mbox{ if } i \geq 2
  \end{align*}
  
  \item % b
  $\{ww^Rw|w \in \Sigma^*\}$ \\
  \\
  Intuition tells us it is not CFL, because we would need more than one stack to match.
  So we try disproving with pumping lemma.
  \begin{align*}
    \mbox{Let } z &= a^n a^n a^n \\
    &= u v^i w x^i y \\
    |vwx| \leq n \\
    |vx| \geq 1 \\
    v,x \mbox{ contains } a's: \\
    \overbrace{\underbrace{a \cdots a}_{uvwxy}}^{n} \overbrace{a \cdots a}^{n} \overbrace{a \cdots a}^{n} \\
    u v^i w x^i y \notin L \mbox{ if } i \geq 2
  \end{align*}
  
  \item % c
  $\{a^i b^j | i \neq j\}$ \\
  \\
  This language is equivalent to $\{a^i b^j | i > j\} \bigcup\, \{a^i b^j | i < j\}$, and
  we know from class that those two are CFL, so under union they should also be.
  We can give a CFG, where $AB$ is for $i > j$, and $CD$ is for $i < j$.
  \begin{align*}
    S &\Rightarrow AB \,|\, CD \\
    A &\Rightarrow aA \,|\, a \\
    B &\Rightarrow aBb \,|\, ab \\
    C &\Rightarrow aCb \,|\, ab \\
    D &\Rightarrow bD \,|\, b
  \end{align*}
  
  \item % d
  $\{a^p|p$ is a prime$\}$ \\
  \\
  Intuition says it is not CFL because we would need to calculate primes.
  If $|\Sigma| = 1$, then both versions of the pumping lemma (CFL and regular languages)
  become the same, and we already proved it in class for the regular language example.
  So we can safely say that this is not CFL.
  
  \item % e
  $\{a^n ww^R a^n | n \geq 0, w \in \{a,b\}^*\}$ \\
  \\
  This one is tricky. PDA's exist for $\{a^n a^n | n \geq 0\}$ and $\{ww^R | w \in \{a,b\}^*\}$
  (we can accept on empty stack for both with non-deterministic PDA), so you would expect this
  language (which is a combination of the two) to also have a PDA and thus be CFL. But first
  we try to disprove with pumping lemma, and if not successful, we try to prove with PDA or CFG.
  \begin{align*}
    \mbox{Let } z &= a^n b^m b^m a^n \\
    &= u v^i w x^i y \\
    |vwx| \leq n \\
    |vx| \geq 1 \\
    v,x \mbox{ contains } a's: \\
    \overbrace{\underbrace{a \cdots a}_{uvwxy}}^{n} \overbrace{b \cdots b}^{m} \overbrace{b \cdots b}^{m} \overbrace{a \cdots a}^{n} \\
    u v^i w x^i y \notin L \mbox{ if } i \geq 2
  \end{align*}
\end{enumerate}

\end{enumerate}

\end{document}
